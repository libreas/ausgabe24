\documentclass[a4paper,
fontsize=11pt,
%headings=small,
oneside,
numbers=noperiodatend,
parskip=half-,
bibliography=totoc,
final
]{scrartcl}

\usepackage{synttree}
\usepackage{graphicx}
\setkeys{Gin}{width=.6\textwidth} %default pics size

\graphicspath{{./plots/}}
\usepackage[ngerman]{babel}
%\usepackage{amsmath}
\usepackage[utf8x]{inputenc}
\usepackage [hyphens]{url}

\usepackage[colorlinks, linkcolor=black,citecolor=black, urlcolor=blue,
breaklinks= true]{hyperref}
\usepackage{breakurl}
\usepackage{booktabs} 
\usepackage[left=2.4cm,right=2.4cm,top=2.3cm,bottom=2cm,includeheadfoot]{geometry}
\usepackage{eurosym}
\usepackage{multirow}
\usepackage[ngerman]{varioref}
\setcapindent{1em}
\renewcommand{\labelitemi}{--}
\usepackage{paralist}
\usepackage{pdfpages}
\usepackage{lscape}
\usepackage{float}
\usepackage{acronym}
\usepackage{eurosym}
\usepackage[babel]{csquotes}
\usepackage{longtable,lscape}
\usepackage{mathpazo}
%\usepackage[flushmargin,ragged]{footmisc} % left align footnote

\urlstyle{same}  % don't use monospace font for urls

\usepackage[fleqn]{amsmath}

%adjust fontsize for part

\usepackage{sectsty}
\partfont{\large}

%Das BibTeX-Zeichen mit \BibTeX setzen:
\def\symbol#1{\char #1\relax}
\def\bsl{{\tt\symbol{'134}}}
\def\BibTeX{{\rm B\kern-.05em{\sc i\kern-.025em b}\kern-.08em
    T\kern-.1667em\lower.7ex\hbox{E}\kern-.125emX}}

\usepackage{fancyhdr}
\fancyhf{}
\pagestyle{fancyplain}
\fancyhead[R]{\thepage}

%meta
\fancyhead[L]{W. Kaiser \\ % author
LIBREAS. Library Ideas, 24 (2014) % journal, issue, volume
\href{http://nbn-resolving.de/urn:nbn:de:kobv:11-100215931}{urn:nbn:de:kobv:11-100215931}} %urn
\fancyfoot[L] {\textit{Creative Commons BY 3.0}} %licence
\fancyfoot[R] {\textit{ISSN: 1860-7950}}

\title{\LARGE{Auf in die Zukunft! Was kommt nach der bücherlosen Bibliothek? Reflexionen und Wahrnehmungsunterschiede zur Rolle von öffentlichen Bibliotheken}} %title
\author{Wolfgang Kaiser} %author

\date{}
\begin{document}

\maketitle
\thispagestyle{fancyplain} 

%abstracts
\begin{abstract}
\small
Die Sehnsucht nach papierlosen und digitalen Bibliotheken ist auch in
Deutschland weit verbreitet. Dabei entsteht der Irrglaube, die alleinige
digitale Ausrichtung der eigenen Bibliothek mache diese zukunftsfähig.
Ökonomisierungstendenzen, ein Mangel an Handeln nach ethischen
Prinzipien, die ständige Messung von Ausleihen und \enquote{Kunden} sind
das Spiegelbild des gegenwärtigen ideologischen Zeitgeistes. Im Artikel
werden vermeintliche Glücksversprechen entlarvt und Alternativen
aufgezeigt. Es wird für eine Öffnung hin zu anderen verwandten
Disziplinen und der Förderung von mehr Vielfalt in der Ausbildung von
Bibliothekaren und Bibliothekarinnen plädiert. Alternative Hinweise und
Anregungen, welche für eine Neubewertung öffentlicher Bibliotheken
eintreten, sind Teil des folgenden Beitrags.
\end{abstract}

\begin{abstract}
\small
The longing for paperless and digital libraries is also very widespread
in Germany as well as in other countries. At the same time there's the
misconception that the digital orientation might be the silver bullet
for their institutions for a sustainable future. Tendencies of
economization, the steady measurement of growth in library loans, and
the notion and perception of clients and customers in libraries are a
mirror image of our current ideological Zeitgeist. The article unmasks
assumed promises of digitalization. It illustrates alternatives for the
implementation of more democratic and participatory library policies.
Furthermore the author pleads for an opening to other disciplines and
the promotion of more diversity within the library field. Alternative
leads and suggestions, which advocate a different evaluation of public
libraries, are part of the following article.
\end{abstract}

\begin{center}\rule{3in}{0.4pt}\end{center}

%body

Die im Herbst 2013 eröffnete papierlose Zweigstellenbibliothek im
texanischen Bexar County, genannt BibloTech\footnote{\url{http://bexarbibliotech.org/}},
sorgt bis heute auch im deutschsprachigen Raum für einen großen
Medienwirbel. Doch die Frage, die sich jeder/jede\footnote{Ich verwende
  das generische Maskulinum an Stellen, an denen ich es für sinnvoll
  erachte, aber an anderen Stellen auch eine gendergerechte
  Schreibweise.} stellt, lautet doch: Ist die Etablierung einer
papierlosen Bibliothek nicht auch hierzulande möglich? Der Blogger Umair
Haque schrieb am 7. Oktober 2013 hierzu folgenden Kommentar auf Twitter:
\enquote{Foodless food, newsless news. And now\ldots{}bookless
libraries}\footnote{\url{https://twitter.com/umairh/status/387459872469438464}}
Er verwies damit wohl unbeabsichtigt auf die typischen Phänomene der
ideologischen Postmoderne, wie sie Slavoj Žižek und Robert Pfaller vor
allem in den westlichen Gesellschaften analysierten. Annette Brüggemann
brachte dies in ihrer Rezension zu Pfallers Buch \enquote{Wofür es sich
zu leben lohnt} auf den Punkt:~

\begin{quote}
\enquote{Wie steht es um unsere vermeintlich hedonistische Kultur, die
aus lauter ‚Non-isms` besteht, wie es der slowenische Philosoph Slavoj
Žižek scharfzüngig formulierte? Kaffee ohne Koffein, Bier ohne Alkohol,
Cola ohne Kalorien, Sahne ohne Fett, Sex ohne Körperkontakt. Das
Paradoxe ist: mit den ‚Non-isms` wird ein Glücksversprechen
verkauft.}\footnote{\url{http://www.deutschlandfunk.de/hymne-an-das-leben.700.de.html?dram:article\_id=84990}}
\end{quote}

Nun ja, Bibliotheken ohne Bücher, das klingt wie Pommes ohne
Ketchup/Majonäse. Wird mit der \enquote{Besinnung} ausschließlich auf
das Digitale nicht ebenso eine Idee beziehungsweise ein
Glücksversprechen verkauft? Und wenn ja, welches? Das eines Glücks durch
Verzicht, durch Optimierung, durch Ersatz?

\section{Ideologie versus Genuss}\label{ideologie-versus-genuss}

Die Zukunft ist schon da! Sie ist papierlos, fleischlos, zuckerfrei,
fettarm, alkoholfrei, nikotinfrei, koffeinfrei und natürlich nachhaltig!
Bitte keine Missionierung und kein Sendungsbewusstsein mehr! Auch ich
lese digital, sehe aber ebenso Vorzüge im Analogen.\footnote{\url{http://www.huffingtonpost.de/andre-wilkens/analog-ist-das-neue-bio\_b\_4793133.html?utm\_hp\_ref=tw}}
Warum diese übertriebene Anbetung und Verherrlichung des Digitalen? Ich
habe großes Verständnis für all jene, die digital leben wollen. Doch es
ist der Grad der Umsetzung und der Beschäftigung mit diesen Themen, der
nicht nur meiner Meinung nach etwas Fanatisches und Beängstigendes
aufweist und an die
\enquote{Wer-nicht-für-uns-ist-der-ist-gegen-uns-Ideologie} erinnert.
Juli Zeh würde diese Haltung und Einstellung als totalitäre Ideologie
bezeichnen.\footnote{\url{http://www.faz.net/aktuell/feuilleton/buecher/rezensionen/belletristik/juli-zehs-neuer-roman-geruchlos-im-hygieneparadies-1774442.html}}
In bestimmten Gruppen, zum Beispiel in Wohngemeinschaften und
Dating-Portalen wollen viele lieber unter Ihresgleichen bleiben und/oder
andere zum Gesundleben bekehren. Differenzerleben beziehungsweise vom
eigenen Absolutheitsanspruch abzuweichen, zu tolerieren und daraus zu
lernen, dass andere Werthaltungen nebeneinander existieren dürfen,
scheint zunehmend weniger möglich zu sein. Der Kölner Psychiater und
Theologe Manfred Lütz beklagte in seinem Buch \enquote{Lebenslust. Wider
die Diät-Sadisten, den Gesundheitswahn und den Fitness-Kult}, dass das
Streben nach Gesundheit einer Ersatzreligion gleicht.\footnote{\url{http://www.welt.de/print/wams/vermischtes/article13773126/Gesundheit-ist-nicht-das-hoechste-Gut.html}}
Manchmal scheint das, wenn es um die Anbetung des Digitalen geht, auch
die Haltung der Medien und Vertreter des Berufsstandes zu treffen. Nun,
ich bin kein digitaler Analphabet und ich wünsche jedem/jeder den Zugang
zum Internet und ausreichend Medienkompetenz. In einer Studie des Pew
Research Center aus Jahr 2013,\footnote{\url{http://libraries.pewinternet.org/2013/02/06/should-libraries-shush/}}
wurden 2.252 US-Amerikaner befragt. Sie wünschten sich vor allem die
vier wichtige Angebote in Bibliotheken: BibliothekarInnen, die ihnen
dabei helfen, Informationen zu finden, Bücher ausleihen zu können, einen
freien Zugang zu Computern und zum Internet. An vierter Stelle folgte
der Wunsch nach ruhigen Lernorten für Kinder und Erwachsene
ebenso.\footnote{\url{http://bibliothekarisch.de/blog/2013/04/24/zum-internationalen-tag-gegen-laerm-how-quiet-should-school-libraries-be/}}
Wie würden die Antworten hierzulande lauten, wenn eine ähnliche Umfrage
durchgeführt werden würde?

Könnten Bibliotheken aber nicht neben der Förderung des Digitalen auch
Orte werden beziehungsweise bleiben, an denen die Entschleunigung und
Kontemplation gefördert wird, um vom täglichen E-Mail- und Online-Stress
abzuschalten? Wie viele gestresste Nutzer/Nutzerinnen und
Bibliothekare/Bibliothekarinnen sind täglich mit einer hohen
Arbeitsdichte konfrontiert und finden keine Ruhe im Hier und Jetzt?

Meines Erachtens könnten sowohl öffentliche Bibliotheken als auch
bestimmte wissenschaftliche Bibliotheken einige Prinzipien des
\enquote{Sabbath Manifesto}\footnote{\url{http://www.sabbathmanifesto.org/}}
wie etwa \enquote{Avoid Technology}, \enquote{Find Silence},
\enquote{Connect With Loved Ones} oder \enquote{Avoid Commerce}
zusätzlich in ihr Aufgabenspektrum mit aufnehmen. Seit wenigen Jahren
erfreut sich der \enquote{Tag zum digitalen Verzicht} in den USA
zunehmend einer größeren Beliebtheit und Anhängerschaft. Das
\enquote{Sabbath Manifesto} hat Gemeinsamkeiten mit dem \enquote{Slow
Movement}, dem \enquote{Slow Food} und dem \enquote{Slow living}. Der
damit verbundende und alljährliche \enquote{National Day of Unplugging}
will Menschen zumindest einen Tag im Jahr davon überzeugen, auf
Computer, Smartphones, Laptops und elektronischen Geräten zu
verzichten.\footnote{\url{http://bibliothekarisch.de/blog/2011/03/04/aus-aktuellem-anlass-was-das-sabbath-manifesto-und-der-heutige-national-day-of-unplugging-fur-uns-heisen-konnten/}}

Man kann sich ein Szenario in naher Zukunft vorstellen, in dem all jene,
welche noch auf Papier lesen, in der Minderheit sind. In diesem würden
dann alle Papierbuchleser ähnlich an den Pranger gestellt, wie derzeit
Raucher von Arbeitskollegen, Freunden und in den Medien ermahnt werden:
\enquote{Gewöhnen Sie sich das ab, denn es schadet der Umwelt und vor
allem dem Regenwald!}

Dann werden mit Sicherheit die letzten analogen Bücherleser, welche es
noch wagen, Bücher aus echtem Papier zu lesen, über die Folgen für das
Waldsterben aufgeklärt und in einer Endlosschleife auf Schritt und Tritt
belehrt werden: \enquote{Erst wenn alle Wälder renaturiert, der letzte
Papierbuchleser bekehrt, der letzte Altbestand entsorgt,\footnote{\url{http://www.sueddeutsche.de/karriere/historische-buecher-wertvolles-kulturgut-im-altpapier-1.554124}}
werdet Ihr merken, dass das Digitale allein kein Glücksversprechen und
Allheilmittel für die Zukunft der Bibliotheken ist!}\footnote{Angelehnt
  an die Weissagung der Cree; vgl.
  \url{http://www.umweltunderinnerung.de/index.php/kapitelseiten/oekologische-zeiten/88-die-schornsteinbesetzer-von-greenpeace}}

In John Christophers Dystopie \enquote{Die Wächter} gelten Bücher (aus
Papier) als \enquote{schmutzige, unhygienische Dinger}, als
\enquote{Fallen für Bakterien.} Wenn Einrichtungen wie BiblioTech die
Leitbilder und Benchmarks für die Meinung des Mainstreams sind, könnten
dann Bücher aus Papier und deren Nutzer sich zunehmend einem
Rechtfertigungsdruck ausgesetzt fühlen und ideologisch als
\enquote{unhygenisch} betrachtet werden? Sind Einrichtungen wie
BiblioTech die sterile, zukunftsweisende und technische Vollendung einer
Bibliothek?\footnote{\url{http://bibliothekarisch.de/blog/2014/01/26/warum-buecherlose-bibliotheken-kein-alleiniges-gluecksversprechen-fuer-die-zukunft-sind/}}

Ähnlich wie Juli Zeh in ihrer Dystopie \enquote{Corpus delicti} die
gegenwärtige Gesundheitsdiktatur\footnote{\url{http://www.deutschlandfunk.de/ein-plaedoyer-gegen-den-gesundheits-und-fitnesswahn.691.de.html?dram:article\_id=56526}}
kritisiert, scheint es sich mit dem Agenda Setting des Fetischisierens
der Digitalisierung und des Lesens von digitalen Büchern zu sein. Dabei
geht es vor allem um die Kombination von Leichtigkeit und Masse: Wir
können alles immer ohne größere materielle Bindung auf einem handlichen
Endgerät abrufen, das zugleich die Erinnerung an die Materialität
darstellt, aber vielleicht auch nur ein Zwischenschritt ist. Fast
scheint es, als ginge es darum, alles was bindet, was wiegt und was
damit auch verpflichtet so weit wie möglich reduziert werden soll und
zugleich doch der Zugang zu allem, was wir wollen, nämlich jeder Facette
kultureller Produktion, permanent gegeben ist. Entworfen wird ein
Schlaraffenland: Wir können konsumieren so viel wir wollen und platzen
doch nie.

Dieses Zukunftsversprechen geht mit dem einher, was Herrmann Rösch auf
der 5. BID-Tagung 2013 in Leipzig Novolatrie nannte. Man bewertet alles
Neue per se als gut.\footnote{\url{http://bibliothekarisch.de/blog/2013/03/23/meine-persoenliche-rueckschau-auf-den-bid-kongress-2013-teil-4/}}
Wir müssen und wollen nur an das Gute des Fortschritts glauben. Wer
skeptisch ist, bremst unseren Weg ins digitale Eden.

Eigentlich ist es erstaunlich, wie sehr die Grenzen dieses Glaubens
nicht gesehen werden. Ein Flachbildschirm schafft keine bessere Welt.
Sie zeigt nur eine, aus der viele Probleme der nicht-virtuellen
Gegenwart einfach herausgerechnet werden. Eine angemessene Aufgabe der
Bibliotheken wäre es, diese wieder hinzuzufügen.

Warum stehen also nicht Aufgaben, wie etwa die soziale Verantwortung,
sowie die Hervorhebung und Analyse von Bildungseffekten im Fokus von
Berichterstattung und Lehre? Warum ist dagegen die Einführung der
Onleihe in einer Stadtbibliothek jeder Lokalzeitung einen Artikel wert?

\section{Marktkonforme (öffentliche) Bibliotheken versus
Post-Wachstumsbibliotheken}\label{marktkonforme-uxf6ffentliche-bibliotheken-versus-post-wachstumsbibliotheken}

\begin{quote}
\enquote{Die kulturelle Dienstleistung Bibliothek darf nicht in den
Haushaltslöchern verschwinden. {[}\ldots{}{]} Unsere Gesellschaft
braucht eine stärkere politische Sicht auf Bibliotheken.} Monika Ziller
am 17. März 2010\footnote{\url{http://www.bibliotheksportal.de/service/nachrichten/archiv/einzelansicht/article/die-kulturelle-dienstleistung-bibliothek-darf-nicht-in-den-haushaltloechern-verschwinden-monika.html}}
\end{quote}

Willkommen in der schönen (neuen) Bibliothekswelt! Sind Sie schon
Premiumkunde oder immer noch ein \enquote{armer} Standardkunde? Die
Ökonomisierung des öffentlichen Raumes macht auch vor Bibliotheken nicht
halt. Wer Bestseller lesen will, bezahlt in einigen öffentlichen
Einrichtungen eine Extrabeitrag, den sich viele Menschen nicht leisten
können beziehungsweise wollen. Die Benutzung der Toiletten der Hamburger
Bücherhallen, ist, wie ich unlängst erfahren durfte, kostenpflichtig.

Marktwirtschaftliche Prinzipien sind in Profitorganisationen der
Normalfall. Nun drängen sie in öffentlich und kommunal finanzierten
Einrichtungen, wie etwa Bibliotheken. Hermann Rösch sieht in dieser
Unterwerfung zugunsten marktwirtschaftlicher Prinzipien das Prinzip der
Gleichbehandlung in Gefahr, wenn Bibliotheksnutzer nach ihrem Einkommen
beurteilt werden.\footnote{\url{http://www.b-u-b.de/chancengleichheit-zur-rolle-bibliothek-in-gesellschaft/}}
Ingo Schulze hatte dies in seiner 2012 gehaltenen Rede \enquote{Unsere
schönen neuen Kleider. Gegen die marktkonforme Demokratie -- für
demokratiekonforme Märkte} zum Ausdruck gebracht:~

\begin{quote}
\enquote{Wenn die Kassen leer sind, muss noch mehr Vermögen privatisiert
werden, müssen Stellen gestrichen und Dienstleistungen privatisiert
werden, müssen Sponsoren gefunden werden, Schwimmbäder und Bibliotheken
geschlossen, die Gebühren in der Musikschule erhöht werden etc. etc. Es
trifft jene, die jeden Euro umdrehen müssen.}\footnote{\url{http://www.ingoschulze.com/rede\_dresden.html}}
\end{quote}

Die Medaille der Chancengleichheit, welche Hermann Rösch forderte, hat
zwei Seiten. Sind das Customer Relationship Management und andere
Theorien aus der Betriebswirtschaft nicht sogar Teil der Lehre an vielen
Fachhochschulen, welche Bibliothekare und Bibliothekarinnen ausbilden?
An welchen Werten und Normen, an welchem Bibliotheksbild orientiert man
sich bei der Gestaltung der Lehrpläne und Berufungen?

Ich bin nicht per se gegen alles, was in diesem Bereich gelehrt wird,
sondern bemerke immer wieder, dass das Wie und Warum zu kurz kommen. Wo
bleibt noch Raum für soziale, pädagogische und interdisziplinäre Themen,
welche der eigentlichen Vielfalt des Berufes mehr Aufmerksamkeit zu Teil
werden lassen? Eigentlich wäre dies der ideale Slogan für öffentliche
Bibliotheken. Ist der Kern der Bibliothek das Bibliotheksmanagement?
Oder ist es nicht eher die Aufgabe, niedrigschwellig Zugänge der
Mitglieder zu bestimmten Aspekten der gesellschaftlichen Teilhabe
abzusichern (Freie Entfaltung der Persönlichkeit, Gleichheit,
Informationsfreiheit et cetera)?

Dass sie inkludierend Grundrechte sichert, ist eines ihrer
Alleinstellungsmerkmale und sollte es auch bleiben. Mediziner haben den
Anspruch, sich an den Eid des Hippokrates zu halten. Heute gibt es das
Genfer Ärztegelöbnis.\footnote{\url{http://www.gesundheit.de/lexika/medizin-lexikon/genfer-aerzte-geloebnis}}
Könnte ein verkürzter IFLA-Ethikkodex auf die deutschen Verhältnisse
übertragen werden und in verkürzte Form eines Gelöbnisses umgewandelt
werden, das bei der \enquote{Vereidigung} zum Bibliothekar/zur
Bibliothekarin zur Pflicht wird? Wie wäre es mit einem Ehrenkodex, um
das Bewusstsein einer Verantwortung für die Gesamtbevölkerung ernster zu
nehmen?

Bei einem Besuch der Bibliotheek Rotterdam erhielt ich 2009 einen
Begrüßungsflyer, in dem zu lesen war, dass diese einen Rückgang an
Ausleihen zu verzeichnen haben, was für die Einrichtung kein Problem
darstellte. Dort wurde schon damals der Abschied von der
bestandsorientierten Bibliotheksarbeit hin zum benutzerorientierten
Arbeiten vollzogen. Das Wachstum an Ausleihen und Beständen spielt in
dieser Einrichtung eine untergeordnete Rolle.\footnote{\url{http://www.bi-international.de/download/file/2009Konf\_Kaiser\_65\_68\_BIT\%201\_2010\%20Heft1-2.pdf}}
Das neue Paradigma wäre eine Post-Wachstumsbibliothek, welche die
Hinwendung zu einer nachhaltigen Entwicklung und deren
sozial-ökologische Faktoren im Mittelpunkt steht.\footnote{\url{http://wirtschaftslexikon.gabler.de/Archiv/576005964/postwachstumsoekonomie-v2.html}}
Die Förderung der Sozialen Nachhaltigkeit durch öffentliche
Bibliotheken\footnote{\url{http://bibliothekarisch.de/blog/2012/06/10/was-heisst-soziale-nachhaltigkeit-fuer-eine-gerechte-stadtbibliotheksentwicklung-ein-plaedoyer-fuer-eine-staerkung-der-sozialen-kohaesion/}}
könnte der Stärkung der sozialen Kohäsion in bestimmten Stadtteilen und
Gemeinden dienen.

Für zahlenorientierte Unterhaltsträger und Sponsoren könnte der Social
Return on Investment (SROI) ein geeignetes Instrument sein. Kommunale
Investitionen werden zu einem erheblichen Teil in sogenannten weichen
Feldern getätigt, also in Bereichen, in denen keine konkreten Zahlen
vorliegen, um Ergebnisse der geleisteten Arbeit zu bewerten. Die
Abteilungen für Bildung und Kultur, unter denen Bibliotheken meist
fallen, haben Aufgaben, die nicht monetär sind.\footnote{\url{https://www.csi.uni-heidelberg.de/kompakt/pdf/CSI\_kompakt\_02\_Social\_Return\_on\_Investment\_Methode.pdf}}
Die SROI-Methode kann dazu dienen, bei unterschiedlichen Akteuren und
Organisationen die Wirkungen sozialer Investitionen zu bestimmen und
Entscheidungen zu begründen.\footnote{\url{http://www.muenster.de/stadt/zuwanderung/pdf/2006\_SROI\_d.pdf}}

Dadurch lassen sich die sozialen Wirkungen, zum Beispiel die
Bildungseffekte, durch bestimmte Programme, Angebote und
Dienstleistungen der jeweiligen öffentlichen Bibliothek transparenter
ermitteln und vermitteln. Über diesen Weg könnte man auch sichtbarer
machen, dass es bei einer Bibliothek nicht darum geht, möglichst viel
Inhalt mit möglichst wenig Aufwand aus dem Bestand in die Nutzerschaft
zu übertragen.

\section{Postdemokratische Einrichtungen versus Partizipation \&
Mitbestimmung}\label{postdemokratische-einrichtungen-versus-partizipation-mitbestimmung}

Der Begriff der Postdemokratie geht auf den britischen
Politikwissenschaftler Colin Crouch zurück, der einen Verlust an
Legitimität von den politischen Akteuren und Institutionen konstatierte.
Das Handeln für das Gemeinwohl tritt dabei in den Hintergrund.
Ohnmachtsgefühle, Entpolitisierung und Politikverdrossenheit sind die
häufigsten Folgen dieses Zustands.\footnote{\url{http://www.bpb.de/apuz/33562/postdemokratie}}
Was heißt das für öffentliche Bibliotheken? Das Beispiel der Stadt
Kassel macht sehr gut deutlich, dass trotz des Bürgerbegehrens
\enquote{Stadtteilbibliotheken erhalten}\footnote{\url{http://www.stadt-kassel.de/aktuelles/rathauswoche/infos/19108/}}
die niedrige Wahlbeteiligung dafür sorgte, dass die Schließungen von
drei Stadtteilbibliotheken am Ende nicht verhindert werden konnten. Der
erfolglose Bürgerentscheid kostete die Stadt 217.000 Euro, wohingegen
die Einsparungen dreier Stadtteilbibliotheken jährlich 360.000 Euro
betrugen.\footnote{\url{http://www.hna.de/lokales/kassel/sonntag-entscheidet-sich-zukunft-bibliotheken-2981371.html}}
Ähnliche Fälle, die zeigen, dass die Bürgerbeteiligung durch Petitionen
und öffentlichkeitswirksame Aktionen oftmals nicht ausreichen, um das
Bibliothekssterben zu verhindern, gibt es auch in Bottrop, Berlin und
anderswo.

Welche Lösung bleibt, wenn dieser Vertrauensverlust in der
repräsentativen Demokratie auch im Bibliotheksbereich Konsequenzen mit
sich bringt? Neben Peter Jobmann, Gerhard Zschau und Heike
Stadler\footnote{\url{http://opus.kobv.de/ubp/volltexte/2011/4924/}}
sind bislang kaum Stimmen im deutschen Bibliothekswesen zu vernehmen,
die Aspekte der Partizipation beziehungsweise der Demokratie(-pädagogik)
im Zusammenhang mit dem Bibliothekssektor thematisieren. Sie forder(t)en
eine Abkehr von einem haupt\-säch\-lich auf Fragen der Verwaltung
gerichteten Bibliothekswesen. Die Orientierung an den Grundwerten der
Demokratie sehen Zschau und Jobmann als einzigen Weg für die Zukunft von
Bibliotheken.\footnote{\url{http://ultrabiblioteka.de/?p=1050}}

In ähnlicher Form äußerte sich hierzu Hermann Rösch auf der 5.
BID-Tagung 2013 in Leipzig, da seiner Meinung nach die
Bestandsorientierung in deutschen Bibliotheken immer noch vom
obrigkeitsstaalichen Handeln geprägt ist.\footnote{\url{http://bibliothekarisch.de/blog/2013/03/23/meine-persoenliche-rueckschau-auf-den-bid-kongress-2013-teil-4/}}
Liegt das an der Mentalität und Kultur in Deutschland? Thomas
Sattelberger, der ehemalige Personalvorstand der Deutschen Telekom AG,
prognostizierte im letzten Jahr, dass die sinnerfüllende Arbeit, die
demokratische Mitbestimmung am Arbeitsplatz und das Postulat der
Vielfalt\footnote{\url{http://www.inqa.de/SharedDocs/PDFs/DE/sattelberger-die-arbeitswelt-von-morgen.pdf?\_\_blob=publicationFile}}
in Zukunft die Arbeitswelt bestimmen wird.

Wie kann es (bibliothekarischen) Lobbyorganisationen, wie etwa dem
Deutschen Bibliotheksverband (DBV), künftig gelingen, auf überregionaler
und nationaler Ebene, ähnlich wie Terre des Hommes oder Robin Wood, in
der breiten Bevölkerung bekannter zu werden und dazu beizutragen, das
mediale Image bibliothekarischer Einrichtungen mitzugestalten? Diese
Lobbyorganisation sollte keinesfalls nur aus BibliothekarInnen und
PolitikerInnen bestehen, sondern aus Menschen unterschiedlicher Berufe
und sozialer Herkünfte.\footnote{\url{http://bibliothekarisch.de/blog/2012/01/06/ein-dritten-sektor-um-das-sterben-oeffentlicher-bibliotheken-und-anderer-kultureinrichtungen-aufzuhalten/}}

\section{Überwachen und Strafen versus moralische Verantwortung
des
Berufsstands}\label{uxfcberwachen-und-strafen-versus-moralische-verantwortung-des-berufsstands}

Für manche mag es die \enquote{Bibliothek des Jahres} sein, für andere
wiederum an die JVA Stammheim (II)\enquote{\footnote{\url{http://www.staatsanzeiger.de/index.php?id=8\&tx\_ttnews{[}backPid{]}=7\&tx\_ttnews{[}tt\_news{]}=280}}
erinnern, obwohl deren Toiletten vermutlich nicht besser überwacht sind
als die der Stuttgarter Stadtbibliothek.\footnote{\url{http://www.stuttgarter-nachrichten.de/inhalt.stadtbibliothek-stuttgart-videoueberwachung-in-toilette.12277b06-ea7d-4d22-a112-baa9712324fc.html}}
Vor kurzem gab es in der Mailingliste Forum ÖB tatsächlich eine
Diskussion (}RFID vs.~Nichtübertragbare Bibliotheksausweise\enquote{),
wie Nutzer und Nutzerinnen von öffentlichen Bibliotheken besser
kontrolliert werden können, da diese den Ausweis des Kindes
beziehungsweise Ehemannes}missbrauchen\enquote{, um eigene Ausleihen zu
tätigen. Wie diese}Vergehen" besser kontrolliert und geahndet werden
können, stieß auf erstaunliches Interesse.

Shaked Spier verwies 2012\footnote{\url{https://drawer20.files.wordpress.com/2012/03/zwischen-bibliothekaren-030412-bd.pdf}}
auf einige weitere Alleinstellungsmerkmale, wie sie zumindest noch auf
die Mehrheit der Bibliotheken zutreffen. Die Privatsphäre und Anonymität
der BibliotheksnutzerInnen wird normalerweise innerhalb der Bibliothek
geschützt. Zu Recht plädierte er dafür, dass Themen wie
Internetüberwachung und Vorratsdatenspeicherung nicht weiter durch
Bibliotheken vernachlässigt werden sollten. Er forderte Stellungnahmen
und eine stärkeren Einsatz für den Kampf um demokratische
Werte.\footnote{Ebda.}

Doch inwiefern sind die ethischen Prinzipien den Bibliothekaren und
Bibliothekarinnen im Wortlaut bekannt? Und inwieweit ist es denn
Vertretern des Berufsstandes in der heutigen Zeit noch möglich, diese zu
leben und danach zu handeln? Stehen diese nicht manchmal auch im
Widerspruch zu Forderungen der kommunalen Unterhaltsträger? Hermann
Rösch wies 2010 zu Recht darauf hin, dass die meisten Bibliotheken in
Deutschland trotz zahlreicher Fortschritte und Liberalisierungen
\enquote{noch sehr stark hierarchisch organisiert} sind.\footnote{\url{http://www.goethe.de/wis/bib/fdk/de6529506.htm}}
Deshalb ist es ein kulturelles \enquote{Problem} und wird sich womöglich
erst verändern, wenn die Generation Baby Boomer und die Generation X in
den Ruhestand verabschiedet werden.

Das E-Book könnte die Überwachung sogar noch ganz anders in die
Bibliotheken bringen. Laut dem Internetsoziologen Stephan Humer wird bei
der Nutzung eines E-Books das Leseverhalten überwacht.\footnote{\url{http://www.3sat.de/page/?source=/nano/gesellschaft/172631/index.html}}
Welche ethische Position vertreten die unterschiedlichen Berufsverbände
hierzu? Andererseits haben drei Viertel der Bundesbürger keine Problem
mit den Überwachungsakti\-vi\-täten der National Security Agency.\footnote{\url{http://www.handelsblatt.com/politik/deutschland/umfrage-nsa-bereitet-bundesbuergern-kaum-sorgen/9019180.html}}
Ein rein mehrheitsorientiertes, trivialdemokratisches Verständnis könnte
nun einfordern, dass sich das vierte Viertel dem zu beugen hat.
Glücklicher\-weise stehen die Grundrechte noch über derartigen
Verschiebungen in der Gewichtung von Einstellungen. Die Frage ist nun,
ob BibliothekarInnen entgegen dem Zeitgeist handeln können und ob sie
imstande sind, öffentlichkeitswirksam ein neues Bewusstsein für diese
Problematik schaffen?

\section{Plädoyer für eine reflexive Wende in Studium und
Ausbildung}\label{pluxe4doyer-fuxfcr-eine-reflexive-wende-in-studium-und-ausbildung}

Ein lang gedienter Bibliothekar, mit dem ich unlängst in Kontakt stand,
beklagte Folgendes: \enquote{Wenn ich mir anschaue was an meiner
ehemaligen Fachhochschule gelehrt wird, dass viel neue IT im Mittelpunkt
steht, viel über Digitalisierung et cetera geredet wird, und andere
Dinge, aber die notwendigen ‚Soft Skills‛ genauso wie die eigentlich
unabdingbaren theoretischen Kenntnisse der kommunalen Verwaltung auf der
Strecke bleiben beziehungsweise ins Praktikum abgeschoben werden.}

Hat er Recht? Gibt es hierzu schon Untersuchungen? Ein Artikel, der im
Januar 2014 im Berliner Tagesspiegel erschien, machte darauf aufmerksam,
dass viele Hochschullehrer bei der Ausbildung die Finanzkrise noch immer
ignorieren.\footnote{\url{http://www.tagesspiegel.de/wirtschaft/volkswirtschaft-im-hoersaal-professoren-wollen-von-der-krise-nichts-wissen/9288094.html}}
BibliothekarInnen sind zwar keine angehenden Ökonomen, aber sie sollten
doch mehr von Ökonomie und Volkswirtschaft verstehen, als nur die
Fähigkeit zu erlernen, Businesspläne zu erstellen und Theorien aus der
profitorientierten Unternehmenswelt auf die Bibliothekswelt zu
übertragen. Welche Effekte und Risiken bringt eigentlich die
Ökonomisierung\footnote{\url{http://www.bertelsmannkritik.de/verwaltung.htm}}von
Bibliotheken mit sich? Wie verändert sich dadurch das eigene Selbst- und
Fremdbild? Hierzu könnten auch Strategien und Lösungsansätze gelehrt
werden, wie finanzschwache Kommunen ihre Bibliotheken erhalten und deren
Wertschätzung in der Bevölkerung und den Medien erhöhen. Die
Gemeingüter- oder Commonstheorie hätte es verdient, in der
bibliothekarischen Lehre und Praxis einen würdigen Platz zu erhalten und
mehr Wertschätzung zu erfahren.\footnote{\url{http://bibliothekarisch.de/blog/2012/01/06/ein-dritten-sektor-um-das-sterben-oeffentlicher-bibliotheken-und-anderer-kultureinrichtungen-aufzuhalten/}}In
vielen Gemeinden gibt es ein Wirtshaussterben und öffentliche
Einrichtungen, wie etwa Schwimmbäder, stehen ebenfalls vor dem Aus. Zur
Rettung wurden beispielsweise Genossenschaften\footnote{\url{https://www.gv-bayern.de/standard/artikel/gasthaeuser-und-brauereien-in-der-rechtsform-eg-1492}}
und Vereine\footnote{\url{http://www.nnz-online.de/00\_nordthueringen/news/news\_lang.php?ArtNr=113999}}
gegründet.

Was können angehende Bibliotheks- und Informationswissenschaftler
praktisch tun, um Bibliotheken vor dem Sterben retten zu lernen?
Studenten der Freien Universität Berlin bieten seit einiger Zeit
alternative Lehrveranstaltungen an, welche die Krise und deren Ursachen
zu erklären versuchen. Welche alternativen Lehrveranstaltungen wären im
Bereich Information und Bibliothek denkbar?

In Deutschland ist aktuell eine Debatte neu entfacht worden, bei der es
um die Parallelgesellschaft Theater geht, in der \enquote{Weiße für
Weiße}\footnote{\url{http://www.br.de/radio/bayern2/sendungen/zuendfunk/kolumnen-sendungen/generator/von-weissen-fuer-weisse-100.html}}
Theater machen und beispielsweise Afrodeutsche aufgrund ihres Aussehens
kaum Rollen als Schauspieler bekommen. Was hat das mit Bibliotheken zu
tun? Die viel geforderte Vielfalt ist unter den Studierenden der
Bibliotheks- und Informationswissenschaften nicht intentional gefördert
worden. Intersektionalität als Untersuchungsgegenstand und Teil der
Lehre hat bislang kaum Eingang in die Bibliothekswelt gefunden.

Katharina Walgenbach definiert Intersektionalität als die
\enquote{sozialen Kategorien wie etwa Gender, Ethnizität, Nation oder
Klasse, die nicht isoliert voneinander konzeptualisiert werden können,
sondern in ihren \enquote{Verwobenheiten} oder \enquote{Überkreuzungen}
(intersections) analysiert werden müs\-sen.}\footnote{\url{http://portal-intersektionalitaet.de/theoriebildung/schluesseltexte/walgenbach-einfuehrung/}}
Entscheidend dabei ist es, die sozialen Ungleichheiten, welche
Chancengleichheit verhindern, zu analysieren und deren Wechselwirkungen
mit der (Nicht-)Nutzung von Bibliotheken und deren mangelnde
\enquote{Barrierefreiheit}, die Mark Terkessidis weiter fasste, als nur
für Menschen mit Behinderung. In Bezug auf Hochschulen, welche spätere
BibliothekarInnen ausbilden, heißt das, dass das dort die
\enquote{Anerkennung der neuen demographischen Vielheit, deren Regeln,
Personal und Strategien} überprüft werden sollten.\footnote{\url{http://www.inklusive-menschenrechte.de/typ/mensch/blog/wp-content/uploads/2010/07/20100430\_iz3w\_terkessidis\_inklusion.pdf}}

Während meines nicht-modularisierten Studiums Bibliotheksmanagement an
der Fachhochschule Potsdam gab es noch die Möglichkeit, ein Nebenfach zu
wählen, das auf den ersten Blick nicht direkt in Verbindung mit der
Bibliothekswelt stand. Studierende hatten die Möglichkeit, Nebenfächer,
wie etwa Sozialpädagogik, Kulturarbeit und Japanologie zu belegen,
wodurch die Grenzen des bibliothekarischen Horizonts überschritten
werden konnte. Vorurteile und Stereotype gegenüber dem Berufsfeld
Bibliothek und Information wurden so frühzeitig überwunden und neue
Ideen für die spätere eigene Arbeit entstanden. Weitere
grenzüberschreitende Impulse stammen aus dem aktuellen Europawahlkampf,
bei dem die CDU dazu aufforderte, die Mobilität der deutschen
Studierenden zu erhöhen, so dass bis 2020 die Hälfte der
Hochschulabsolventen im Ausland studiert haben sollte.\footnote{\url{http://www.welt.de/politik/deutschland/article124441555/CDU-will-deutsche-Jugend-auf-Wanderschaft-schicken.html}}
Vor zwei Jahren bot ich einer deutschen Fachhochschule eine mögliche
ungarische Partnerhochschule in Pécs (Ungarn) an, da ein mir bekannter
Professor daran Interesse hatte. Eine Reaktion der zuständigen Person
bleibt bis heute aus und ich schäme mich im Nachhinein dafür, dass diese
Anfrage schlicht und einfach ignoriert wurde. Wie viele angehende
Bibliotheks- und Informationswissenschaftler haben mindestens ein
Semester im Ausland verbracht? Die Förderung der Mehrsprachigkeit und
die Verbesserung der Vergleichbarkeit des bibliotheks- und
informationswissenschaftlichen Studiums, würden die vielfach geforderte
Mobilität innerhalb Europas erhöhen und ein bessere Zusammenwachsen der
Community ermöglichen. Drei Tage BOBCATSSS im Jahr sind eindeutig zu
wenig für eine Verständigung und einen Dialog von angehenden
europäischen Bibliothekaren und Bibliothekarinnen!

\section{Fazit}\label{fazit}

Wir brauchen mehr Utopien, die nicht nur in der Gesellschaft, sondern
auch in den Bibliotheks- und Informationswissenschaften gedacht, gelebt
und gelehrt werden könnten. Das sollte sich vor allem in Ausbildung und
Studium widerspiegeln. Diese wären in der Lage,
Fehlentwicklungen\footnote{\url{http://www.greenpeace-magazin.de/magazin/archiv/1-14/brauchen-menschen-utopien/}}
entgegenzutreten und Wege aus der gegenwärtigen (vermeintlich)
ausweglosen Ökonom\-isierungsfalle aufzuzeigen. Die Einbeziehung und
Erweiterung des Gemeingüterbegriffs auf öffentliche
Bibliotheken\footnote{\url{http://www.gemeingut.org/uber-uns/grundsatze/}}
wäre ein erster Fortschritt, um eine Neudefinition der Rolle von
öffent\-lichen Bibliotheken einzuleiten. Die aktuelle Diskussion um das
Für und Wider eines Neubaus der Zentral- und Landesbibliothek in
Berlin\footnote{\url{http://www.tagesspiegel.de/meinung/lesermeinung/lesermeinung-kein-mensch-braucht-die-landesbibliothek/9500198.html}}
steht exemplarisch dafür, die Herausforderung eine Neubewertung von
öffentlichen Bibliotheken zum Anlass zu nehmen und diese im aktuellen
Mediendiskurs auf die Agenda zu setzen. Eine Stadt wie Berlin, welche
weiterhin von anderen Bundesländern subventioniert wird und in der jeder
5. Einwohner von Armut bedroht\footnote{\url{http://www.rbb-online.de/politik/beitrag/2013/12/Armut-Berlin-Brandenburg-Bericht.html}}
ist, wäre ein ideales Labor, um Utopien zu verwirklichen. Bislang wurden
in Berlin seit dem Fall der Mauer sehr viele Bibliotheksschließungen und
Kaputtsparmaßnahmen\footnote{\url{http://www.tagesspiegel.de/berlin/bibliotheken-in-berlin-die-haelfte-der-buechereien-ist-geschlossen/9350980.html}}
in die Tat umgesetzt. Öffentlichen Bibliotheken sollten künftig zum
Grundbestand eines jeden Bezirks zählen und ebenso wie Straßen und
Schulen keine freiwillige Aufgabe mehr sein.

Bislang wurde der Wert von Bibliotheken eher rein ökonomisch nach Zahlen
und dem Return on Investment gemessen. Ich plädiere daher dafür,
Gemeinden und Städten, welche sich bereits ihrer öffentlichen
Bibliotheken aufgrund von Sparzwängen entledigten, mit solchen zu
vergleichen, welche über eine hervorragende Bibliotheksinfrastruktur
verfügen. Die sozialen Folgen einer fehlenden Bibliotheksinfrastruktur
können verheerend für den sozialen Frieden und das künftige
Bildungspotential sein. Die Lebensqualität und das Wohlbefinden von
Bürgern misst sich auch daran, welcher Wert öffentlichen Bibliotheken
beigemessen wird. Die soziale Spaltung verläuft entlang der Kommunen und
Bundesländer, in den die Bibliotheksinfrastruktur nur noch rudimentär
vorhanden ist. \enquote{In Afrika sagt man, wenn ein alter Mann stirbt,
verschwindet eine Bibliothek.}\footnote{\url{https://www.unric.org/html/german/senioren/presse/2.pdf}}
Wie reagiert man eigentlich im Bibliotheksentwicklungsland
Deutschland,\footnote{\url{http://www.faz.net/aktuell/rhein-main/kommentar-bibliothekarisches-entwicklungsland-11056832.html}}
wenn eine Bibliothek gebaut wird beziehungsweise verschwindet? Häufig
werden Bedenken geäußert oder es herrscht Gleichgültigkeit. Ich plädiere
daher für ein Umdenken, das die Existenz und den Unterhalt von
Bibliotheken in Schulen und Gemeinden zu einer Pflichtaufgabe macht.
Frei nach Winston Churchill muss sich eine sogenannte Bildungsrepublik
Deutschland nicht daran bemessen lassen, wie viel Geld Sie für die
Digitalisierung von Büchern ausgibt, sondern was ihr eine
\enquote{echte} und nachhaltige Verwirklichung von Chancengleichheit
durch öffentliche Bibliotheken wert ist.

\section{Literaturverzeichnis}\label{literaturverzeichnis}

AFP: Umfrage: NSA bereitet Bundesbürgern kaum Sorgen. In: Handelsblatt
vom 02.11.2013. Online zugänglich
unter:~\url{http://www.handelsblatt.com/politik/deutschland/umfrage-nsa-bereitet-bundesbuergern-kaum-sorgen/9019180.html},
vom 19.02.2014

AG Du bist Bertelsmann (2009): Privatisierung der Kommunalen Verwaltung.
Online zugänglich
unter:\url{http://www.bertelsmannkritik.de/verwaltung.htm}, vom
19.02.2014

Annan, Kofi A: Altern und Entwicklung: Eine Gesellschaft für alle
Altersgruppen schaffen! Zweite Weltversammlung zur Frage des Alterns,
Madrid, Spanien, 8 -- 13. April 2002. Online zugänglich unter:
\url{https://www.unric.org/html/german/senioren/presse/2.pdf}

BiblioTech. Bexar County Digital Library. Online zugänglich unter:~
\url{http://bexarbibliotech.org/}, vom 19.02.2014~

Brüggemann, Annette (Rezension): Hymne an das Leben. Robert Pfaller
(2011): \enquote{Wofür es sich zu leben lohnt} - Elemente
materialistischer Philosophie. S. Fischer Wissenschaft. Beitrag vom
16.03.2011. Online zugänglich
unter:~\url{http://www.deutschlandfunk.de/hymne-an-das-leben.700.de.html?dram:article_id=84990}

Centrum für soziale Investitionen und Innovationen der Universität
Heidelberg: Erfolge messen und belegen Transparenz schaffen mit der
`Social Return on Investment`-Methode. In: CSI Kompakt 02, November
2012, Online zugänglich unter:~
\url{https://www.csi.uni-heidelberg.de/kompakt/pdf/CSI_kompakt_02_Social_Return_on_Investment_Methode.pdf},
vom 19.02.2014

Deutscher Bibliotheksverband (2010): \enquote{Die kulturelle
Dienstleistung Bibliothek darf nicht in den Haushaltslöchern
verschwinden.} Monika Ziller, Direktorin der Stadtbibliothek Heilbronn,
neue dbv-Vorsitzende. Pressemitteilung des Deutschen Bibliotheksverbands
vom Mittwoch, 24. März 2010. Online zugänglich unter:\\
\href{http://www.bibliotheksportal.de/service/nachrichten/archiv/einzelansicht/article/die-kulturelle-dienstleistung-bibliothek-darf-nicht-in-den-haushaltloechern-verschwinden-monika.html}{http://www.bibliotheksportal.de/service/nachrichten/archiv/einzelansicht/article/die-kulturelle-dienstleistung-bibliothek-darf-nicht-in-den-haushaltloechern-verschwinden-monika.html},
vom 19.02.2014

Dobberke, Cay/Schönball, Ralf: Bibliotheken in Berlin**Die Hälfte der
Büchereien ist geschlossen. In: Tagesspiegel vom 18.01.2014. Online
zugänglich
unter:~\url{http://www.tagesspiegel.de/berlin/bibliotheken-in-berlin-die-haelfte-der-buechereien-ist-geschlossen/9350980.html}

Donbib (2014): Das große Wort: Demokratie -- hier mal für Bibliotheken.
In: Ultra Biblioteka vom 13. Januar 2013. Online zugänglich unter:~
\url{http://ultrabiblioteka.de/2013/01/13/das-grose-wort-demokratie-hier-mal-fur-bibliotheken/},
vom 19.02.2014

3Sat: E-Books: Lesen und gelesen werden vom 10. Oktober 2014~ Online
zugänglich unter:~
\url{http://www.3sat.de/page/?source=/nano/gesellschaft/172631/index.html},
vom 19.02.2014

Frewer, Andreas/Rothaar, Markus: Das Recht des Menschen und die Medizin
60 Jahre Genfer Gelöbnis und Allgemeine Erklärung der Menschenrechte*.
In: MRM ---MenschenRechtsMagazin Heft 2/2008, S. 139 -- 141. Online
zugänglich unter:~ \url{http://opus.kobv.de/ubp/volltexte/2009/3601/},
vom 19.02.2014

\newpage

Gemeingut in BürgerInnenhand e.V.: Grundsätze. Online zugänglich
unter: \\
\href{http://www.gemeingut.org/uber-uns/grundsatze}{http://www.gemeingut.org/uber-uns/grundsatze}, vom
19.02.2014

Genossenschaftsverband Bayern (2013): Gasthäuser und Brauereien in der
Rechtsform eG. Online zugänglich
unter: \\
~\href{https://www.gv-bayern.de/standard/artikel/gasthaeuser-und-brauereien-in-der-rechtsform-eg-149}{https://www.gv-bayern.de/standard/artikel/gasthaeuser-und-brauereien-in-der-rechtsform-eg-149},
vom 19.02.2014

Geyer, Christian: Juli Zehs neuer Roman Geruchlos im Hygieneparadies.
In: FAZ vom 01.03.2009. Online zugänglich unter:~
\url{http://www.faz.net/aktuell/feuilleton/buecher/rezensionen/belletristik/juli-zehs-neuer-roman-geruchlos-im-hygieneparadies-1774442.html?printPagedArticle=true},
vom 19.02.2014

Geilhufe, Martin (2012): Die Schornsteinbesetzer von Greenpeace. Online
zugänglich unter:~ \\
\href{http://umweltunderinnerung.de/index.php/kapitelseiten/oekologische-zeiten/88-die-schornsteinbesetzer-von-greenpeace/66-die-schornsteinbesetzer-von-greenpeace}{http://umweltunderinnerung.de/index.php/kapitelseiten/oekologische-zeiten/88-die-schornstein\-besetzer-von-greenpeace/66-die-schorn\-steinbesetzer-von-greenpeace},
vom 19.02.2014~

Giersberg, Dagmar (2010): Diskussion um ethische Grundsätze für
Bibliothekare. Online zugänglich
unter:~\url{http://www.goethe.de/wis/bib/fdk/de6529506.htm}, vom
19.02.2014

Grossmann, Alexander: Lesermeinung: \enquote{Kein Mensch braucht die
Landesbibliothek}. In:Tagesspiegel vom 18.02.2014. Online zugänglich
unter:~\url{http://www.tagesspiegel.de/meinung/lesermeinung/lesermeinung-kein-mensch-braucht-die-landesbibliothek/9500198.html},
vom 19.02.2014

Haque, Umair: Foodless food, newsless news. And now\ldots{}bookless
libraries.8. Oktober 2013, 08:10 a.m. Tweet. Online zugänglich unter:~
\url{https://twitter.com/umairh/status/387459872469438464}, vom
19.02.2014

Interview mit Mark Terkessidis über sein neues Buch
\enquote{Interkultur}. In: IZ3W, Mai/Juni 2010. Online zugänglich unter:
\url{http://www.inklusive-menschenrechte.de/typ/mensch/blog/wp-content/uploads/2010/07/20100430_iz3w_terkessidis_inklusion.pdf},
vom 19.02.2014

Interview mit Juli Zeh, 8. Januar 2012, Ein Plädoyer gegen den
Gesundheits- und Fitnesswahn.\enquote{Wir schenken Ihnen Zeit} - VIII:
Schriftstellerin über Krisenhysterie und~ Online zugänglich unter:~
\url{http://www.deutschlandfunk.de/ein-plaedoyer-gegen-den-gesundheits-und-fitnesswahn.691.de.html?dram:article_id=56526}``,
vom 19.02.2014

Interview mit Richard Saage: Brauchen Menschen Utopien? In: Greenpeace
Magazin, 1/2014 Online zugänglich unter:~
\url{http://www.greenpeace-magazin.de/magazin/archiv/1-14/brauchen-menschen-utopien},
vom 19.02.2014

Kaiser, Wolfgang (2012): Ein dritter Sektor um das Sterben öffentlicher
Bibliotheken und anderer Kultureinrichtungen aufzuhalten. In:
Bibliothekarisch.de vom~ 06.01.2012. Online zugänglich
unter:~\url{http://bibliothekarisch.de/blog/2012/01/06/ein-dritten-sektor-um-das-sterben-oeffentlicher-bibliotheken-und-anderer-kultureinrichtungen-aufzuhalten/},
vom 19.02.2014

Kaiser, Wolfgang: Multikulturelle Bibliotheksarbeit. Bericht über eine
Tagung in den Niederlanden im November 2009. In: B.I.T.online 13 (2010)
Nr. 1. Online zugänglich unter:~
\url{http://www.b-i-t-online.de/heft/2010-01/reportage3}, vom 19.02.2014

Kaiser, Wolfgang (2011): Aus aktuellem Anlass: Was das \enquote{Sabbath
Manifesto} und der heutige \enquote{National Day of Unplugging} mit
\enquote{uns} zu tun haben könnten. In: Bibliothekarisch.de vom
04.02.2011. Online zugänglich
unter:\url{http://bibliothekarisch.de/blog/2011/03/04/aus-aktuellem-anlass-was-das-sabbath-manifesto-und-der-heutige-national-day-of-unplugging-fur-uns-heisen-konnten/},
vom 19.02.2014

Kaiser, Wolfgang (2013): Meine persönliche Rückschau auf den
BID-Kongress 2013 (Teil 4). In: Bibliothekarisch.de vom 23.03.2013.
Online verfügbar unter:~
\url{http://bibliothekarisch.de/blog/2013/03/23/meine-persoenliche-rueckschau-auf-den-bid-kongress-2013-teil-4/}, \\
vom 19.02.2014

Kaiser, Wolfgang (2014): Warum bücherlose Bibliotheken kein alleiniges
Glücksversprechen für die Zukunft sind. In: Bibliothekarisch.de vom
26.01.2014. Online zugänglich unter:~
\\
\url{http://bibliothekarisch.de/blog/2014/01/26/warum-buecherlose-bibliotheken-kein-alleiniges-gluecksversprechen-fuer-die-zukunft-sind/},
vom 19.02.2014

Kaiser, Wolfgang (2012): Was heißt soziale Nachhaltigkeit für eine
gerechte Stadtbibliotheksentwicklung? Ein Plädoyer für eine Stärkung der
sozialen Kohäsion. In: Bibliothekarisch.de vom 10.06.2012. Online
zugänglich
unter:\url{http://bibliothekarisch.de/blog/2012/06/10/was-heisst-soziale-nachhaltigkeit-fuer-eine-gerechte-stadtbibliotheksentwicklung-ein-plaedoyer-fuer-eine-staerkung-der-sozialen-kohaesion/}

Kaiser, Wolfgang (2013): Drohnenflug in der New York Public Library. In:
Bibliothekarisch.de vom 10.11.2013. Online zugänglich
unter:
\\
~\url{http://bibliothekarisch.de/blog/2013/11/10/drohnenflug-in-der-new-york-public-library/}, \\
vom 19.02.2014

Kaiser, Wolfgang (2013): Zum Internationalen Tag gegen Lärm:
\enquote{How Quiet Should Libraries~Be?} In: Bibliothekarisch.de vom
24.04.2013 Online zugänglich unter~ \\
\href{http://bibliothekarisch.de/blog/2013/04/24/zum-internationalen-tag\%0Dgegen-laerm-how-quiet-should-school-libraries-be/}{http://bibliothekarisch.de/blog/2013/04/24/zum-internationalen-tag\%0Dgegen-laerm-how-quiet-should-school-libraries-be/}, 
vom 19.02.2014

Khamis, Kammis: Parallelgesellschaft Theater \enquote{Von Weißen für
Weiße}. Online zugänglich
unter:~\url{http://www.br.de/radio/bayern2/sendungen/zuendfunk/kolumnen-sendungen/generator/von-weissen-fuer-weisse-100.html},
vom 19.02.2014~

Lossau, Norbert: Gesundheit ist nicht das höchste Gut. In: Welt am
Sonntag vom 18.11.2011. Online zugänglich \\
unter:~\url{http://www.welt.de/print/wams/vermischtes/article13773126/Gesundheit-ist-nicht-das-hoechste-Gut.html},
vom 19.02.2014

Marion: Wert Wissen In: Der Karfiol vom 6. Jan. 2013, 15:17 Uhr Online
zugänglich unter:\\~ \url{http://derkarfiol.de/?p=226}, vom 19.02.2014~

Neuhaus, Carla: Volkswirtschaft im Hörsaal Professoren wollen von der
Krise nichts wissen. In: Tagesspiegel vom 5. Januar 2014. Online
zugänglich\\
unter:~\url{http://www.tagesspiegel.de/wirtschaft/volkswirtschaft-im-hoersaal-professoren-wollen-von-der-krise-nichts-wissen/9288094.html},
vom 19.02.2014

Obermaier Frederick (dpa): Historische Bücher: Wertvolles Kulturgut im
Altpapier? In: SZ vom 17.10.2010. \\ Online zugänglich
unter:~\url{http://www.sueddeutsche.de/karriere/historische-buecher-wertvolles-kulturgut-im-altpapier-1.554124},
vom 19.02.2014

O. Verf.: Roche Lexikon Medizin. Genfer Ärztegelöbnis. Online zugänglich
unter:~\url{http://www.gesundheit.de/lexika/medizin-lexikon/genfer-aerzte},
vom 19.02.2014~

O. Verf.: Zu wenig abgegebene Stimmen. Büchereien vor dem Aus:
Bürgerentscheid war erfolglos -- zu wenig abgegebene Stimmen. In:
Hessische/Niedersächsische Allgemeine vom 30.06.2013. Online zugänglich
unter:
\url{http://www.hna.de/lokales/kassel/sonntag-entscheidet-sich-zukunft-bibliotheken-2981371.html},
vom 19.02.2014

O. Verf.: Postdemokratie. In: APuZ, 1-2/2011. Online zugänglich unter:~
\url{http://www.bpb.de/shop/zeitschriften/apuz/33561/postdemokratie},
vom 19.02.2014

O. Verf.: Hochschulen: CDU will deutsche Jugend auf Wanderschaft
schicken. In: Die Welt vom 1. Februar 2014. \\ Online zugänglich unter:~
\url{http://www.welt.de/politik/deutschland/article124441555/CDU-will-deutsche-Jugend-auf-Wanderschaft-schicken.html},
vom 19.02.2014

O. Verf.: Kluft zwischen Arm und Reich wächst -- Jeder Fünfte in der
Region ist armutsgefährdet. \\ Online zugänglich
unter: \\ \url{http://www.rbb-online.de/politik/beitrag/2013/12/Armut-Berlin-Brandenburg-Bericht.html},
vom 19.02.2014

O. Verf.: Sabbath Manifesto. Online zugänglich unter:~
\url{http://www.sabbathmanifesto.org/}, vom 19.02.2014

O. Verf.: Schwimmbadverein in Sicht. In: Neue Nordhäuser Zeitung vom
26.07.2012. Online zugänglich
unter:~\url{http://www.nnz-online.de/00\_nordthueringen/news/news_lang.php?ArtNr=113999},
vom 19.02.2014

Riebsamen, Hans: Kommentar: Bibliothekarisches Entwicklungsland. In: FAZ
vom 12.10.2010. \\ Online zugänglich
unter: \\ ~\url{http://www.faz.net/aktuell/rhein-main/kommentar-bibliothekarischesentwicklungsland-11056832.html},
vom 19.02.2014

Rösch, Hermann 2014): Chancengleichheit -- ein Thema für Bibliotheken?
Zur Rolle der Bibliothek in der Gesellschaft, In: BuB 66, (2014) 02, S.
110 -- 113. \\
\url{http://www.b-u-b.de/chancengleichheit-zur-rolle-bibliothek-in-gesellschaft/},
vom 19.02.2014

Spier, Shaked: Zwischen Bibliothekaren und Bücherwürmern. Über das
(fehlende) soziale Engagement der Information Community. In:
Bibliotheksdienst 46. Jg. (2012), H. 3/4, S. 171-181. Online zugänglich
unter:~
\url{https://drawer20.files.wordpress.com/2012/03/zwischen-bibliothekaren-030412-bd.pdf},
vom 19.02.2014

Springer Gabler Verlag (Herausgeber), Gabler Wirtschaftslexikon,
Stichwort: Postwachstums\-ökonomie, Online zugänglich
unter: \\ ~\url{http://wirtschaftslexikon.gabler.de/Archiv/576005964/postwachstumsoekonomie-v2.html},
vom 19.02.2014

Sattelberger, Thomas: Die Arbeitswelt von morgen. In: Personalmagazin
05/13, S. 28-29. Online zugänglich unter:~
\url{http://www.inqa.de/SharedDocs/PDFs/DE/sattelberger-die-arbeitswelt-von-morgen.pdf},
vom 19.02.2014

Schulze, Ingo: \enquote{Unsere schönen neuen Kleider. Gegen die
marktkonforme Demokratie -- für demokratiekonforme Märkte.} Dresdner
Rede vom 26. 02. 2012. Online zugänglich unter:~
\url{http://www.ingoschulze.com/rede_dresden.html}, vom 19.02.2014

Schwarz, Konstantin: Stadtbibliothek Stuttgart: Videoüberwachung in der
Toilette. Stuttgarter Nachrichten vom 23.01.2014. Online zugänglich
unter: \\ ~\url{http://www.stuttgarter-nachrichten.de/inhalt.stadtbibliothek-stuttgart-videoueberwachung-in-toilette.12277b06-ea7d-4d22-a112-baa9712324fc.html},
vom 19.02.2014

Stadt Kassel (2013): Stadtteilbibliotheken: Bürger stimmen am 30. Juni
ab. Online zugänglich
unter:~\url{http://www.stadt-kassel.de/aktuelles/rathauswoche/infos/19108/"},
vom 19.02.2014

Stadler, Heike: Partizipation -- Bibliothek. \\ Online verfügbar unter:~
\url{http://bibpartizipation.wordpress.com/}, vom 19.02.2014

Walgenbach, Katharina (2012): Intersektionalität - eine Einführung. .
Online verfügbar unter:~\url{http://portal-intersektionalitaet.de} , vom 19.02.2014

Wilkens, Andreas (2014): Analog ist das neue Bio. \\ Online zugänglich
unter:~
\url{http://www.huffingtonpost.de/andre-wilkens/analog-ist-das-neue-bio_b_4793133.html},
vom 19.02.2014

Zickuhr, Kathryn (2013): Should libraries shush? In: Pew Research Center
vom 6. Februar 2013. Online zugänglich
unter:\url{http://libraries.pewinternet.org/2013/02/06/should-libraries-shush/},
vom 19.02.2014

%autor
\begin{center}\rule{3in}{0.4pt}\end{center}

\textbf{Wolfgang Kaiser}. Diplom-Bibliothekar. Tätig als Pädagogischer
Mitarbeiter in der Außenstelle Ingolstadt des Deutschen
Erwachsenen-Bildungswerks gemeinnützige GmbH. Zu seinen
Forschungsschwerpunkten zählen Fragen der Erwachsenenbildung und
außerschulischen Jugendbildung (in Bibliotheken), der Diversität, der
sozialen Gerechtigkeit und zum Vergleichenden Bibliothekswesen.

\end{document}
